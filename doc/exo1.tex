\documentclass[a4paper,12pts]{article}

\usepackage[utf8]{inputenc}
\usepackage{hyperref}
\usepackage{minted}

\title{TD : descente de gradient}
\date{ }

\begin{document}
\maketitle

\section{Exercices introductifs (non évalués)}

\begin{itemize}
\item Comprendre et implémenter le code suivant et observer le comportement de
  l'algorithme en faisant varier la valeur du pas de gradient (\verb+step_size+)
\begin{minted}{python}
def optimise_univariate(step_size=0.4,max_epochs=30):
    """
    Optimizes f(x) = (x-3)^2 (strictly convex)
    """    
    x = 0
    for e in range(max_epochs):
        x  -= step_size * 2*(x-3) #f'(x) = 2(x-3)
        obj = (x-3)**2
        print(x,obj)
    return x
\end{minted}
\item Comprendre et implémenter le code suivant et observer le comportement de
  l'algorithme en faisant varier la valeur du pas de gradient
  (\verb+step_size+)
\begin{minted}{python}
def optimise_bivariate(step_size=0.4,max_epochs=30):
    """
    Optimizes f(x1,x2) = x1^2+x2^2+2x1+8x2 (strictly convex)
    """    
    (x1,x2) = (0,0)
    for e in range(max_epochs):
        obj = x1**2+x2**2+2*x1+8*x2
        print((x1,x2),obj)
        x1,x2 = (x1-step_size*(2*x1+2),x2-step_size*(2*x2+8))

    obj = x1**2+x2**2+2*x1+8*x2
    print((x1,x2),obj)
    return (x1,x2)
\end{minted}
\item Lire le code de la classe \verb+LogisticModel+ dans le fichier
  \verb+numericGD.py+. Vérifiez que vous comprenenez l'implémentation de la méthode \verb+train+
\end{itemize}


\section{Exercices notés (à faire en binome)}

Télécharger le corpus
\href{https://gforge.inria.fr/frs/?group_id=3597}{\tt Sequoia} (v6.0) et travailler avec le fichier {\tt sequoia-corpus.np$\_$conll}
\begin{enumerate}
\item Faire une fonction {\tt split(nomfichier)} qui lit un corpus et
  écrit un corpus d'entrainement, un corpus de développement et un
  corpus de test.
\item Faire une fonction {\tt read$\_$corpus(nomfichier)} qui lit les données d'un
  corpus à partir d'un nom de fichier et qui renvoie un jeu de données
  d'entrainement pour votre classifieur.
\item  Faire une classe {\tt AvgPerceptron}
avec au moins une fonction appelée\\ {\tt train(self,dataset,step$\_$size=0.1,max$\_$epochs)}
qui implémente une
fonction d'entrainement d'un perceptron moyenné multiclasse pour
prédire les tags des mots de votre corpus ainsi qu'une fonction
{\tt test(self,dataset)} qui renvoie l'exactitude de votre classifieur sur
le jeu de données passé en paramètres.
\end{enumerate}
\paragraph{Contrainte} Vous utiliserez la classe
\href{https://github.com/bencrabbe/parsing-at-diderot/blob/master/SparseWeightVector.py}{\tt
  SparseWeightVector.py} comme support pour implémenter le vecteur de
poids du perceptron. Vous pouvez consulter le fichier 
\href{https://github.com/bencrabbe/parsing-at-diderot/blob/master/Multiclass.py}{\tt
  Multiclass.py} comme source d'inspiration.

%\paragraph{Discussion} Sur base de tests réalisés à l'aide de votre
%implémentation, illustrez et comparez brièvement les propriétés des courbes d'apprentissage de votre perceptron
%moyenné et de la version non moyennée. Selon vous qu'est-ce qui
%explique les différences ?


\section{Procédure de notation}
On attend le code qui répond aux exercices (1,2,3) ci-dessus
enregistré dans un fichier nommé {\tt exo1-nombinomes.py} 
La procédure de notation sera partiellement automatisée et exécutera (au moins) la séquence d'instructions suivantes.
%https://github.com/bencrabbe/parsing-at-diderot/blob/master/Multiclass.py
\begin{verbatim}
split(sequoia-corpus.np_conll)
trainc = read_corpus(sequoia-corpus.np_conll.train)
testc  = read_corpus(sequoia-corpus.np_conll.test)
p = AvgPerceptron()
p.train(trainc)
print(p.test(testc))
\end{verbatim}
\end{document}